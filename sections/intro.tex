\section{Introduction}

\begin{frame}
    \frametitle{Presentation outline}
    
    \begin{itemize}
        \item Motivation and Objectives
        \item interoperability between CUDA and OpenACC
        \item Basics of Object Orientation and Modern Fortran
        \item OpenACC for OOP
        \item Conclusions
    \end{itemize}
    
\end{frame}

\begin{frame}
	\frametitle{Motivation and Objectives}
	\begin{columns}
		\column{0.5\textwidth}
		\begin{block}{Motivation}
			\begin{itemize}
				\item CASE software (Alya, SOD2D) steadily adopting OpenACC
				\item Fortran still relevant in scientific comunity
				\item Modern Fortran can help clean up code and improve maintainability
				\item Still some friction in adapting GPU code to OOP concepts
				\item Usage of CUDA in external libraries makes interoperability relevant
			\end{itemize}
		\end{block}
		\column{0.5\textwidth}
		\begin{block}{Objectives}
			\begin{itemize}
				\item Explore OOP concepts in Modern Fortran and C++
				\item Demonstrate OpenACC strategies for OOP code
				\item Analyze performance implications and imposed limitations
				\item Showcase interoperability between CUDA and OpenACC
				\item Provide practical examples and best practices
			\end{itemize}
		\end{block}
	\end{columns}
\end{frame}

\begin{frame}
	\frametitle{Why GPUs/OpenACC?}
	\begin{columns}
		\column{0.5\textwidth}
		\begin{itemize}
			\item Allows exploiting massive parallelism at task level with SIMT model
			\item Modern GPUs can provide better performance per watt when well used
			\item OpenACC provides a simpler approach to GPU programming
			\item For most numerical applications, ACC is virtually as fast as CUDA
			\item ACC available for both Fortran and C/C++, interoperability with CUDA
			\item Most HPC systems and hyperscalers moving towards GPU-based architectures
		\end{itemize}
		\column{0.5\textwidth}
		\begin{figure}
			\centering
			\includegraphics[width=0.9\textwidth]{images/palpatine.jpg}
		\end{figure}
	\end{columns}
\end{frame}