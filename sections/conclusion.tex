\section{Closing thoughts}

\begin{frame}
    \frametitle{Conclusions: accuracy and robustness}
    \begin{itemize}
        \item Compressible TGV validation demonstrates SEM-ENVIT model to be a low-dissipation TVD scheme
        \item Both the Explicit and IMEX schemes have shown excellent converge to the reference solution
        \item Entropy viscosity has been shown to work as intended, stabilizing discontinuities without smearing turbulent structures
        \item Explicit scheme suffers from limited CFL condition on supersonic cases, further justifying the IMEX scheme
        \item Cylinder and CRM-HL demonstrate SOD2D's capability to simulate complex geometries and complex flows, demonstrate need for IMEX
    \end{itemize}
\end{frame}

\begin{frame}
    \frametitle{Conclusions: overall performance}
    \begin{itemize}
        \item Kernels have been shown to be efficient on GPUs for high-order elements
        \item Explicit vs. IMEX: dependent on the CFL limit and PCG behaviour, assuming both stable
        \item Impact of system hardware on performance has been demonstrated
        \item Importance of FP32 arithmetics usage on GPUs addressed
        \item SOD2D scalability has been demonstrated on a series of important systems
        \item Importance of proper RDMA capability in modern systems made clear
        \item SOD2D + MN5 ACC is an excellent pairing for high-performance simulations
    \end{itemize}
\end{frame}

\begin{frame}
    \frametitle{Future work}
    \begin{itemize}
        \item Test different time schemes: LS-RK and implicit variations (DIRK, ESDIRK...)
        \item Test communication strategies in MN5, as well as benefits of NCCL library usage
        \item Introduction of combustion models and chemical reactions
        \item Explore possibility of using SEM tetrahedral elements
        \item Allow for non-conformal meshes aiming at AMR
    \end{itemize}
\end{frame}